% SPDX-FileCopyrightText: 2024 Maxim Urschumzew <mxmurw@determi.io>
% SPDX-FileCopyrightText: 2024 Miëtek Bak
%
% SPDX-License-Identifier: MIT

\documentclass{scrartcl}
\usepackage{amsmath}
\usepackage{amssymb}
\usepackage{amsthm}
\usepackage{bm}
\usepackage{enumitem}
\usepackage{graphicx}
\usepackage{mathpartir}
\usepackage[numbers]{natbib}
\usepackage{stmaryrd}
\usepackage{tikz}
\usepackage[hyphens]{url}
\usepackage{xcolor}

% use microtype for advanced typography
\usepackage[activate={true,nocompatibility},
            final,
            tracking=true,
            kerning=true,
            spacing=true,
            factor=1100,
            stretch=10,
            shrink=10]{microtype}
\microtypecontext{spacing=nonfrench}
\SetTracking{encoding={*}, shape=sc}{0}
\urlstyle{same}
\definecolor{urlblue}{rgb}{0,0.2,0.8}

% use hyperref last to avoid problems
\usepackage{hyperref}
\hypersetup{
  allcolors=urlblue,
  bookmarksnumbered=false,
  bookmarksopen=false,
  breaklinks=true,
  colorlinks,
  pdfborder={0 0 0},
  pdfborderstyle={/S/U/W 1}
}

%%%%%%%%%%%%%%%%%%%%%%%%%%%%%%%%%%%%%%%%%%%%%%%%%%%%%%%%%%%%%%%%%%%%%%%%%%%%%%%%
\theoremstyle{definition}
\newtheorem{definition}{Definition}
\newtheorem{example}{Example}
\theoremstyle{plain}
\newtheorem{theorem}{Theorem}
\newtheorem{corollary}{Corollary}
\newtheorem{conjecture}{Conjecture}

\renewcommand{\square}%
  {\tikz{\draw (0,0) rectangle (0.2cm,0.2cm);}}

\renewcommand{\circle}%
  {\tikz{\draw (0,0) circle [radius=0.1cm];}}

\renewcommand{\triangle}%
  {\tikz{\draw (0,0) -- (0.2cm,0) -- (0.1cm,0.2cm) -- (0,0);}}

\newcommand{\primitive}[1]{\textsf{\textbf{#1}}}
\newcommand{\MTTM}{MTT${}_{\mathcal{M}}$}
\newcommand{\ChorMTT}{Chor${}_{\textrm{MTT}}$}
\newcommand{\ProcMTT}{Proc${}_{\textrm{MTT}}$}

%%%%%%%%%%%%%%%%%%%%%%%%%%%%%%%%%%%%%%%%%%%%%%%%%%%%%%%%%%%%%%%%%%%%%%%%%%%%%%%%
\title{Kami: A modal type theory for~distributed~systems}
\author{Maxim Urschumzew \and Miëtek Bak}
\date{\today}
\begin{document}
\maketitle
\begin{abstract}
  \noindent
  We present Kami, a programming language for distributed systems based on
  Modal Type Theory\cite{gratzer2023syntax}. In particular, we describe a mode
  theory that is suitable for distributed programming, and show how the
  primitives of Chor$\lambda$\cite{cruz2022functional}, a choreographic
  programming language, can be recovered in Kami. Finally, we sketch a
  categorical semantics that gives rise to a generic compilation procedure for
  Kami.
\end{abstract}

%%%%%%%%%%%%%%%%%%%%%%%%%%%%%%%%%%%%%%%%%%%%%%%%%%%%%%%%%%%%%%%%%%%%%%%%%%%%%%%%
\section{Introduction}
Modalities provide a way to extend a type theory with domain-specific type
constructors, which can be used to track runtime properties of code directly in
the type system. For example, modalities have been used to represent
unevaluated code for metaprogramming and staged
computation\cite{davies2001modal}, and to encode the notion of location in
ML5\cite{murphy2008modal}, a language for distributed systems. In fact, the
\texttt{at} modality of ML5 serves the same purpose as the location annotations
in choreographic programming languages\cite{cruz2022functional,
giallorenzo2024choral}, and it stands to reason that existing research on
modalities could provide some stepping stones for the design of type systems
for choreographic programming languages.

In this paper, we explore how Chor$\lambda$\cite{cruz2022functional}, a
functional choreographic programming language, can be based on top of a simply
typed lambda calculus (STLC) with modalities, a variant of Modal Type
Theory\cite{gratzer2023syntax}. This requires the introduction of two new
concepts: common knowledge between multiple participants, and local knowledge
of global computations.

%%%%%%%%%%%%%%%%%%%%%%%%%%%%%%%%%%%%%%%%%%%%%%%%%%%%%%%%%%%%%%%%%%%%%%%%%%%%%%%%
\section{Preliminaries}
\subsection{\texorpdfstring{Chor$\lambda$}{ChorLambda}}
Chor$\lambda$\cite{cruz2022functional} is a functional choreographic
programming language introduced by Cruz-Filipe et~al., and an extension of STLC
with location annotations. For example, $t : \textsf{Bool} @ r$ means that $t$
evaluates to a value of type \textsf{Bool} located at role $r$. The language
has two communication primitives: \primitive{com} for communicating data and
\primitive{select} for communicating choice of branching.

%%%%%%%%%%%%%%%%%%%%%%%%%%%%%%%%%%%%%%%%%%%%%%%%%%%%%%%%%%%%%%%%%%%%%%%%%%%%%%%%
\subsection{Simply typed modal type theory}
Modal Type Theory (MTT)\cite{gratzer2023syntax}, introduced by Gratzer in his
PhD dissertation, is a framework for constructing modal type theories, with
Martin-L\"{of} type theory at its core. The framework is parametrized by a
system of modalities specified in the form of a \emph{mode theory}. MTT
includes so-called ``full-spectrum'' dependent types, but we restrict ourselves
to a simply typed fragment of MTT.
\begin{definition}[{\cite[Chapter 6.1.1]{gratzer2023syntax}}]
  A \emph{mode theory} is given by the following data:
  \begin{itemize}
  \item
    A set of \emph{modes} $M$. Every mode $m \in M$ instantiates a distinct
    copy of an extension of STLC.
  \item
    For each pair of modes $m, n \in M$, a set $m \to n$ of \emph{modalities}
    between the modes. A modality $\mu : m \to n$ allows us to use types and
    terms of mode $m$ in mode $n$. There might be multiple distinct modalities
    $\mu , \nu : m \to n$.
  \item
    For every pair of modalities $\mu, \nu \in m \to n$ with matching domains
    and codomains, a set $\mu \Rightarrow \nu$ of \emph{transformations}
    between the modalities. A transformation $\alpha : \mu \Rightarrow \nu$
    allows us to convert types and terms from under the modality $\mu$ to under
    the modality $\nu$. There might be multiple distinct transformations
    $\alpha, \beta : \mu \Rightarrow \nu$.
  \end{itemize}
  Concretely, this data forms a 2-category, in the sense that identity
  modalities (of type $\textsf{id}_m : m \to m$) and identity transformations
  (of type $\textsf{id}_\mu : \mu \Rightarrow \mu $) exist, modalities and
  transformations can be composed (denoted by $\mu \fatsemi \nu$ and $\alpha
  \fatsemi \beta$ respectively) if the domains and codomains match, and
  identity and associativity laws hold as expected\cite{licata2016adjoint}.
\end{definition}
\begin{definition}[{\cite[following Chapter 6.2]{gratzer2023syntax}}]
  Let $\mathcal{M}$ be a mode theory. A \emph{simply typed modal type theory}
  \MTTM{} is given by $|M|$ copies of an extension of STLC, combined as
  follows: For each mode $m \in M$, let $\Gamma\in\textsf{Ctx}_m$, $A \in
  \textsf{Type}_m$, $t \in \textsf{Term}_m$, and $\Gamma \vdash_m t : A$ denote
  the contexts, types, terms, and typing judgements of the $m$th copy of an
  extension of STLC respectively.
  
  The rules of \MTTM{} are displayed in figures \ref{fig:mtt_type},
  \ref{fig:mtt_ctx}, \ref{fig:mtt_var}, and \ref{fig:mtt_term}.
\end{definition}

%%%%%%%%%%%%%%%%%%%%%%%%%%%%%%%%%%%%%%%%%%%%%%%%%%%%%%%%%%%%%%%%%%%%%%%%%%%%%%%%
% Types
\begin{figure}
  \centering
  \begin{mathpar}
    \inferrule*[Lab=Mod-Form]
    {
      A \in \textsf{Type}_m
      \\
      \mu : m \to n
    }
    {\langle A | \mu \rangle \in \textsf{Type}_n}

    \inferrule*[Lab=Fun-Form]
    {
      A \in \textsf{Type}_m
      \\
      \mu : m \to n
      \\
      B \in \textsf{Type}_n
    }
    {(A | \mu) \to B \in \textsf{Type}_n}
    \\
    \inferrule*[Lab=Prod-Form]
    {
      A \in \textsf{Type}_m
      \\
      B \in \textsf{Type}_m
    }
    {A \times B \in \textsf{Type}_m}

    \inferrule*[Lab=Sum-Form]
    {
      A \in \textsf{Type}_m
      \\
      B \in \textsf{Type}_m
    }
    {A + B \in \textsf{Type}_m}
  \end{mathpar}
  \caption{Types of \MTTM{}}
  \label{fig:mtt_type}
\end{figure}

%%%%%%%%%%%%%%%%%%%%%%%%%%%%%%%%%%%%%%%%%%%%%%%%%%%%%%%%%%%%%%%%%%%%%%%%%%%%%%%%
\subsubsection*{Types}
Types of \MTTM{} (figure \ref{fig:mtt_type}) mirror the standard types of STLC,
with the following differences:
\begin{itemize}
\item
  A \emph{modal type} $\langle A | \mu \rangle \in \textsf{Type}_n$ is formed by
  lifting a type $A \in \textsf{Type}_m$ to mode $n$ using a modality $\mu : m
  \to n$.
\item
  A \emph{modal function type} $(A | \mu) \to B \in \textsf{Type}_n$ is formed
  for each type $A : \textsf{Type}_m$, $B : \textsf{Type}_n$, and modality $\mu
  : m \to n$. This type is a convenience that behaves as the standard function
  type composed with a modal type. We recover the standard function type using
  the identity modality $\textsf{id}_m : m \to m$.
\end{itemize}

%%%%%%%%%%%%%%%%%%%%%%%%%%%%%%%%%%%%%%%%%%%%%%%%%%%%%%%%%%%%%%%%%%%%%%%%%%%%%%%%
% Contexts
\begin{figure}
  \centering
  \begin{mathpar}
    \inferrule*[Lab=Ctx-Empty]
    {\textrm{ }}
    {\cdot \in \textsf{Ctx}_m}

    \inferrule*[Lab=Ctx-Ext]
    {
      \Gamma \in \textsf{Ctx}_n
      \\
      A \in \textsf{Type}_m
      \\
      \mu : m \to n
    }
    {\Gamma.(A|\mu) \in \textsf{Ctx}_n}
    
    \inferrule*[Lab=Ctx-Restr]
    {
      \Gamma \in \textsf{Ctx}_n
      \\
      \mu : m \to n
    }
    {\Gamma.\{\mu\} \in \textsf{Ctx}_m}
  \end{mathpar}
  \caption{Contexts of \MTTM{}}
  \label{fig:mtt_ctx}
\end{figure}
  
%%%%%%%%%%%%%%%%%%%%%%%%%%%%%%%%%%%%%%%%%%%%%%%%%%%%%%%%%%%%%%%%%%%%%%%%%%%%%%%%
\subsubsection*{Contexts}
Contexts of \MTTM{} (figure \ref{fig:mtt_ctx}) are defined as follows:
\begin{itemize}
\item
  The empty context rule \textsc{Ctx-Empty}.
\item
  The context extension rule \textsc{Ctx-Ext} says that all types in a context
  $\Gamma \in \textsf{Ctx}_m$ must exist at a mode $m$, but they may originate
  at a different mode $n$ by way of a modality $\mu : n \to m$. This means that
  an assumption in a context $\Gamma$ is of the form $(A | \mu) \in \Gamma$,
  where $A$ is a type, $n$ is an arbitrary mode, and $\mu : n \to m$ is a
  modality that brings the assumption into mode $m$. If the type originates at
  the current mode, then $\mu = \textsf{id}_m : m \to m$.
\item
  The context restriction rule \textsc{Ctx-Restr} allows us to restrict the use
  of types to those under a particular modality. That is, starting with a
  context $\Gamma \in \textsf{Ctx}_n$ and a pair of modalities $\mu, \nu : m
  \to n$, the restricted context $\Gamma.\{\nu\}\in\textsf{Ctx}_m$ only allows
  referring to the assumption $(A | \mu) \in \Gamma$ given a transformation
  $\alpha : \mu \Rightarrow \nu$. This is ensured by the variable rule
  \textsc{Var} (figure \ref{fig:mtt_term}).
\end{itemize}

%%%%%%%%%%%%%%%%%%%%%%%%%%%%%%%%%%%%%%%%%%%%%%%%%%%%%%%%%%%%%%%%%%%%%%%%%%%%%%%%
% Variables
\begin{figure}
  \centering
  \begin{mathpar}
    \inferrule*[Lab=Var-Zero]
    {\ }
    {\primitive{zero} : (A|\mu) \in_{\textsf{id}_n} \Gamma.(A | \mu)}

    \inferrule*[Lab=Var-Suc]
    {
      x:(A|\mu) \in_{\nu} \Gamma
    }
    {\primitive{suc }x : (A|\mu) \in_{\nu} \Gamma.(B | \eta)}

    \inferrule*[Lab=Var-Restr]
    {x:(A|\mu) \in_{\nu} \Gamma}
    {x:(A|\mu) \in_{\eta \fatsemi \nu} \Gamma.\{\eta\}}
  \end{mathpar}
  \caption{Variables (de Bruijn indices) of \MTTM{}}
  \label{fig:mtt_var}
\end{figure}

%%%%%%%%%%%%%%%%%%%%%%%%%%%%%%%%%%%%%%%%%%%%%%%%%%%%%%%%%%%%%%%%%%%%%%%%%%%%%%%%
\subsubsection*{Terms}
Terms of \MTTM{} (figure \ref{fig:mtt_term}) differ from terms of STLC as
follows:
\begin{itemize}
\item
  There are additional terms for introduction and elimination of modal type.
  The introduction rule \textsc{Mod-Intro} enforces that terms of a modal type
  $\langle A | \mu \rangle$ may only depend on variables that are themselves
  under the $\mu$ modality. The elimination rule \textsc{Mod-Elim} allows us to
  assume $x : (A|\mu)$ in order to eliminate a value of such a modal type.
\item
  The terms for introduction and elimination of function type reflect that the
  type is endowed with an additional modality.
\item
  All other terms are mode-local and any modalities that occur in the terms are
  identities.
\end{itemize}

%%%%%%%%%%%%%%%%%%%%%%%%%%%%%%%%%%%%%%%%%%%%%%%%%%%%%%%%%%%%%%%%%%%%%%%%%%%%%%%%
% Terms
\begin{figure}[h]
  \centering
  \begin{mathpar}
    \inferrule*[Lab=Var]
    {
      \mu, \nu : m \to n
      \\
      \alpha : \mu \Rightarrow \nu
      \\
      x:(A|\mu) \in_\nu \Gamma
    }
    {\Gamma \vdash_n x^\alpha : A}
    \\
    \inferrule*[Lab=Mod-Intro]
    {
      \mu : m \to n
      \\
      \Gamma.\{\mu\} \vdash_m t : A
    }
    {\Gamma \vdash_n \primitive{mod}_\mu\  t : \langle A | \mu \rangle}
    \\
    \inferrule*[Lab=Mod-Elim]
    {
      \mu : m \to n
      \\
      \nu : n \to o
      \\
      \Gamma.\{\nu\} \vdash_n s : \langle A | \mu \rangle
      \\
      \Gamma.(x : A|\mu \fatsemi \nu) \vdash_o t : B
    }
    {
      \Gamma \vdash_o \primitive{let}_\nu\:\primitive{mod}_\mu\  x \bm{\leftarrow} s \primitive{ in }
        t : B
    }
    \\
    \inferrule*[Lab=Fun-Intro]
    {
      \mu : m \to n
      \\
      \Gamma.(x : A|\mu) \vdash_n t : B
    }
    {\Gamma \vdash_n \bm{\lambda\,} x \bm{\,.\,} t : (A | \mu) \to B}

    \inferrule*[Lab=Fun-Elim]
    {
      \mu : m \to n
      \\
      \Gamma \vdash_n t : (A | \mu) \to B
      \\
      \Gamma \vdash_n s : A
    }
    {\Gamma \vdash_n t\,s : B}
    \\
    \inferrule*[Lab=Prod-Intro]
    {
      \Gamma \vdash_m s : A
      \\
      \Gamma \vdash_m t : B
    }
    {\Gamma \vdash_m s \bm{\,,\,} t : A \times B}

    \inferrule*[Lab=Prod-Elim${}_1$]
    {\Gamma \vdash_m t : A \times B}
    {\Gamma \vdash_m \primitive{fst } t : A}

    \inferrule*[Lab=Prod-Elim${}_2$]
    {\Gamma \vdash_m t : A \times B}
    {\Gamma \vdash_m \primitive{snd } t : B}
    \\
    \inferrule*[Lab=Sum-Intro${}_1$]
    {\Gamma \vdash_m t : A}
    {\Gamma \vdash_m \primitive{left } t : A + B}

    \inferrule*[Lab=Sum-Intro${}_2$]
    {\Gamma \vdash_m t : B}
    {\Gamma \vdash_m \primitive{right } t : A + B}

    \inferrule*[Lab=Sum-Elim]
    {\Gamma \vdash_m s : A + B
      \\
      \Gamma.(x : A|\textsf{id}_m) \vdash_m t : C
      \\
      \Gamma.(y : B|\textsf{id}_m) \vdash_m u : C
    }
    {\Gamma \vdash_m \primitive{case } s \primitive{ of}
      \,\bm{\{}\, \primitive{left } x \bm{\,\mapsto\,} t \bm{\,;\,}
      \primitive{right } y \bm{\,\mapsto\,} u : C \,\bm{\}}
    }
  \end{mathpar}
  \caption{Terms of \MTTM{}}
  \label{fig:mtt_term}
\end{figure}

%%%%%%%%%%%%%%%%%%%%%%%%%%%%%%%%%%%%%%%%%%%%%%%%%%%%%%%%%%%%%%%%%%%%%%%%%%%%%%%%
\section{Choreographic programming with MTT}
The first, essential feature of modalities in MTT is that they restrict code
availability: When constructing a term of type $\langle A | \mu \rangle$, only
variables that are themselves under a $\mu$ modality can be used. This leads us
naturally to the idea that we can use MTT for a distributed system with a set
of participating roles $\rho$ by introducing modalities $@r$ for each role $r
\in \rho$. The type $\langle A | @r \rangle$ then is interpreted as data of
type $A$, located at role $r$.

The second feature of MTT is that transformations between modalities can be
introduced in a controlled way. In order to allow a term at role $r$ to be
transformed into a term at role $s$, we simply have to introduce a
transformation $\tau_{r,s} : @r \Rightarrow @s$ between the corresponding
modalities. These transformations can be chosen freely: we might disallow
communications between some roles and include multiple channels between some
others.

%%%%%%%%%%%%%%%%%%%%%%%%%%%%%%%%%%%%%%%%%%%%%%%%%%%%%%%%%%%%%%%%%%%%%%%%%%%%%%%%
\subsection{Mode theory for choreographic programming}
However, a mode theory with only $@$-modalities and transformations
$\tau_{r,s}$ is not enough to recover the full expressive power of
Chor$\lambda$. In particular, expressing the \textbf{select} operator, which is
used to notify other roles about decisions that happened locally at role $r$,
requires the following additional features:
\begin{enumerate}
\item
  We need modalities expressing the fact that some data is common knowledge
  between multiple roles. We do so by allowing arbitrary conjunctions $r_1
  \land \ldots \land r_k$ in the modality. For example, the type $\langle
  \mathbb{N} | @(r\land s) \rangle$ expresses the fact that there is a natural
  number that is known by both roles $r$ and $s$.
\item
  We need roles to reference data that is about to be sent to other roles. For
  this we use a new modality $\square$, which allows roles to reference global
  choreographies locally. That is, if $A$ is a global choreography type, then
  $t : \langle A | \square \rangle$ is a local term, containing a quoted
  representation of such a choreography.
\end{enumerate}
We define our mode theory with $@$ and $\square$ modalities and common
knowledge locations as follows:
\begin{definition}
Let $\rho$ be a set of roles and $\textrm{Loc}_\rho$ be the freely generated
meet-semilattice on $\rho$. The mode theory $\mathcal{M}^\rho_{\textrm{Chor}}$
is defined as follows:
\begin{itemize}
\item
  There are two modes: $\circle$ and $\triangle$. The global mode $\circle$
  represents the global perspective on a choreography, encompassing
  computations and data occuring at all roles. The local mode $\triangle$
  represents the perspective of a single location (which might be the
  conjunction of multiple roles) participating in the choreography.
\item
  For each location $u \in \textrm{Loc}_\rho$ there is a modality $@u :
  \triangle \to \circle$. Each of these modalities represents a different way
  of how a local computation can be embedded in the global system. Concretely,
  $@u$ expresses that the computation exists at location $u$.
\item
  An additional modality $\square : \circle \to \triangle$, allowing global
  computations to be referenced locally.
\item
  For each location $u \in \textrm{Loc}_\rho$, a transformation
  $\textsf{quote}_u : \textsf{id}_{\triangle} \Rightarrow ({}@ u \fatsemi
  \square)$. This transformation allows a role to quote a local term to be
  evaluated by an arbitrary role $u$. It happens in local mode ($\triangle$)
  and its interpretation involves no communication.
\item
  For each location $v \in \textrm{Loc}_\rho$, a transformation
  $\textsf{eval}_v : (\square \fatsemi {}@ v) \Rightarrow
  \textsf{id}_{\circle}$. This transformation is the only one involving
  communication between roles. It describes that a global choreography,
  available in quoted form at location $v$, can be scheduled to be executed by
  all involved roles.
\item
  For each pair of locations $u, v \in \textrm{Loc}_\rho$ with $u \leq v$, a
  transformation $\textsf{narrow}_{u,v} : @u \Rightarrow @v$.
\item
  Additional equalities governing the interactions of the transformations. For
  instance, composing $\textsf{quote}_u$ and $\textsf{eval}_u$ for the same
  role $u$ is equal to the identity transformation since it represents a
  role communicating with itself.
\end{itemize}
\end{definition}
\begin{corollary}
  In {\upshape$\mathcal{M}^{\rho}_{\textrm{Chor}}$}, it is possible to recover
  a transformation $\tau_{u,v} : @u \Rightarrow @v$ by composing
  {\upshape$\textsf{quote}_v$} and {\upshape$\textsf{eval}_u$}. Additionally, a
  role communicating with itself is equal to the identity transformation
  {\upshape$\tau_{u,u} = \textsf{id}_{@u}$}.
\end{corollary}

%%%%%%%%%%%%%%%%%%%%%%%%%%%%%%%%%%%%%%%%%%%%%%%%%%%%%%%%%%%%%%%%%%%%%%%%%%%%%%%%
\section{Kami: An MTT based language for choreographic programming}
Our mode theory expresses the interactions between participating roles in a
distributed system, but standard MTT instantiated with
$\mathcal{M}^{\rho}_{\textrm{Chor}}$ is not suitable to be used as a
choreographic programming language. The problem is that there is no notion of
deferred transformations: transformations are always pushed down the syntax
tree as far as possible and only recorded at the variables. This means that in
order to control the communication behaviour of terms, we need to introduce a
dedicated term for not yet executed communications. In our semantics only
$\textsf{eval}_v$ involves communications, so we simply add a dedicated term
representing it. Its typing and reduction rules are displayed in figure
\ref{fig:mtt_leteval}.

\begin{figure}
  \centering
  \begin{mathpar}
    \inferrule*[Lab=Let-Eval]
    {
      \nu : \circle \to m
      \\
      \Gamma .\{\nu\} \vdash_{\circle} s : \langle A | \square \fatsemi @u \rangle
      \\
      \Gamma.(x : A|\nu) \vdash_m t : B
    }
    {
      \Gamma \vdash_m \primitive{let}_\nu\:\primitive{eval}_u\ x \bm{\leftarrow} s
      \primitive{ in } t : B
    }
    \\
    \inferrule*[Lab=Let-Eval-$\beta$]
    {
      \textrm{ }
    }
    {
      \Gamma \vdash_m \primitive{let}_\nu\:\primitive{eval}_u\ x \bm{\leftarrow} s
      \primitive{ in } t \ \Longrightarrow\  \Gamma \vdash_m
      \primitive{let}_\nu\:\primitive{mod}_{(\square \fatsemi @u)}\ y \bm{\leftarrow} s \primitive{ in } t
      [ y^{\textsf{eval}_u} / x ]
    }
  \end{mathpar}
  \caption{Typing and reduction rules for deferred transformations}
  \label{fig:mtt_leteval}
\end{figure}

\begin{definition}
  Let Chor${}_{\textrm{MTT}}$ be the type theory obtained by:
  \begin{enumerate}
  \item
    Instantiating simply typed MTT with $\mathcal{M}^{\rho}_{\textrm{Chor}}$.
  \item
    Extending it with the \textsc{Let-Eval} rule.
  \item
    Restricting the reduction relation analogously to how Chor$\lambda$
    restricts the reduction relation of STLC.
  \end{enumerate}
\end{definition}

%%%%%%%%%%%%%%%%%%%%%%%%%%%%%%%%%%%%%%%%%%%%%%%%%%%%%%%%%%%%%%%%%%%%%%%%%%%%%%%%
\subsection{Informal semantics and relation with
  \texorpdfstring{Chor$\lambda$}{ChorLambda}}
In order to allow for out-of-order execution of of independent processes, we
can use the same machinery as Chor$\lambda$: restricting reduction rules to
ensure that communications between a pair of processes always occur in the same
order and an additional rewriting relation that allows for independent
processes to evaluate their terms independently.

The expressivity of Chor$\lambda$ arises from the two primitives involved in
process interaction: $\primitive{com}$ and $\primitive{select}$. Communication
is easily reproduced in \ChorMTT{} by using $\textsf{quote}$ and
$\textsf{eval}$.
\begin{example}
  In \ChorMTT, we can define a function $\textsf{com}_{A,u,v} : \langle A | @u
  \rangle \to \langle A | @v \rangle$ for each local type $A$ and pair of
  locations $u,v$.
  \begin{align*}
    \textsf{com}_{A,u,v}\ x =
    \ & \primitive{let}\:\primitive{mod}_{@u}\ y \bm{\leftarrow} x\\[-3pt]
      & \primitive{let}\:\primitive{eval}_u\ z \bm{\leftarrow} \primitive{mod}_{@v}\ (\primitive{mod}_{(\square \fatsemi @u)}\ 
        y^{\textsf{quote}_v \,\circledast\, \textsf{id}_{@u}})\\[-3pt]
      & \primitive{in } z^{\textsf{id}_{\textsf{id}_{@v}}}
  \end{align*}
\end{example}
\noindent
Selection in Chor$\lambda$ works as follows: a process at role $r$ chooses its
future behaviour based on locally available data and afterwards communicates
its choice using \primitive{select} statements to those processes that need to
be aware of this choice. In \ChorMTT{} the same functionality is available, but
has to be stated in reverse order. First the required data for choosing future
behaviour is communicated from $r$ to all roles that need to be aware of this
choice. After receiving the data, all relevant processes synchronously decide
their future behaviour.
\begin{example}
  Let $A, B$ be local types, and $Z$ a global type. A function of the following
  type can be derived in \ChorMTT:
  \[
    \textsf{choice}_{A,B,Z,r} : \langle A + B | @r \rangle
    \to (\langle A | @r \rangle \to Z)
    \to (\langle B | @r \rangle \to Z)
    \to Z
  \]
  Note that since $Z$ is global, a value $z : Z$ is a choreography involving
  possibly all roles\footnote{For the sake of brevity our mode theory
  $\mathcal{M}^{\rho}_{\textrm{Chor}}$ as defined in this paper does not track
  which roles are actually involved in a given term of global type. It can be
  done though by extending the mode theory with a family of global modes.}. The
  semantics of $\textsf{choice}$ is: Given the knowledge of $A + B$ at role
  $r$, a global behaviour $z : Z$ can be chosen for all roles, with the
  additional knowledge of either the value $a : A$ or $b : B$ at role $r$. The
  function can be implemented in such a way that only the information regarding
  which branch is going to be chosen is communicated from $r$ to other
  processes, the actual value of $a$ or $b$ stays at $r$.

  The interaction of the $\square$ and $@r$ modalities are key to the
  definition of choice. Gratzer calls a function that allows induction ``from
  under a modality'' a \emph{crisp induction principle}\cite{gratzer2023syntax}.
\end{example}

%%%%%%%%%%%%%%%%%%%%%%%%%%%%%%%%%%%%%%%%%%%%%%%%%%%%%%%%%%%%%%%%%%%%%%%%%%%%%%%%
\subsection{Categorical semantics}
Following Gratzer, a categorical model for MTT$_{\mathcal{M}}$ is given by a
functor $F : \mathcal{M}^{\textsf{coop}} \to \textbf{Cat}$, with additional
conditions ensuring that $F$ supports all type formers and their terms. This
definition entails, for each modality $\mu : m \to n$, a functor $F(\mu) : F(n)
\to F(m)$ modeling the \emph{contravariant} context restriction operation
${-}.\{\mu\} : \textsf{Ctx}_n \to \textsf{Ctx}_m$. Additionally, the existence
of modal types in Gratzer's model implies the existence of a further
\emph{covariant} functor $M_\mu : F(m) \to F(n)$, such that $F(\mu)$ and
$M_\mu$ are adjoint in an appropriate sense\footnote{In the dependently
  typed MTT of Gratzer the exact statement is that $M_\mu$ is a \emph{dependent
  right adjoint}, but we can simplify the condition somewhat in our simply typed
  case.}.

While such a generic Model of MTT is required for Gratzer's goals, our intended
semantics, in particular the fact that we want to compile Kami programs into
real-world executables leads us to considering a less general, but more
specifically useful class of models.

In the following we give the definition of our special class of models and
sketch how MTT${}_{\mathcal{M}^\rho_{\textrm{Chor}}}$, i.e., the underlying
type theory of Kami can be interpreted in them.
\begin{definition}
  Let $\mathcal{M}$ be a mode theory. A \emph{covariant model} for
  simply typed MTT$_{\mathcal{M}}$ is given by the following data:
  \begin{enumerate}
  \item
    A category $\mathcal{C}$ representing the compilation target.
  \item
    Closure of $\mathcal{C}$ under all type and term formers of MTT.
  \item
    A (covariant) 1-functor $G : \mathcal{M} \to \textbf{Cat}$ modeling the
    semantics of individual modes, and the modal types between them.
  \item
    A family of 1-functors $\iota_m : G(m) \to \mathcal{C}$, where $m \in
    \mathcal{M}$, describing how the category $G(m)$ is represented in the
    compilation target category.
  \item
    For each transformation $\alpha : \mu \Rightarrow \nu \in \mathcal{M}$, a
    natural transformation $\tau_\alpha : \iota_m \circ G(\mu) \Rightarrow
    \iota_m \circ G(\nu)$ in the target category.
  \end{enumerate}
\end{definition}
\noindent
Such a covariant model is a special case of a contravariant model in the sense
of Gratzer; we claim:
\begin{conjecture}
  A covariant model of MTT$_{\mathcal{M}}$ can be assembled into a
  contravariant model, by freely adjoining context restriction operators.
\end{conjecture}
\noindent
For our concrete use-case, we intend to obtain a covariant model as follows:
Let $\rho$ be a finite set of roles and let $\textbf{STLC}$ be the syntax
category of STLC with sum types. Viewing the set $\rho$ as a discrete category,
we denote by $\textbf{STLC}^\rho$ the functor category $\rho \to
\textbf{STLC}$. That is, an object $(\Gamma_i)_{i\in\rho}$ is a $\rho$-indexed
family of $\textbf{STLC}$-contexts and a morphism $(\Gamma_i)_{i\in\rho} \to
(\Delta_i)_{i\in\rho}$ is given by a $\rho$-indexed family of substitutions
$(\sigma_i : \Gamma_i \to \Delta_i)_{i\in\rho}$. The intuition is that
$\textbf{STLC}^\rho$ describes the category of $\rho$ processes running
independently of each other, with no way to interact. To further add
synchronous communication, we freely adjoin arrows axiomatizing such. To
properly express these arrows, we need the following definition:
\begin{definition}
  Let $i\in\rho$ be a role, define
  \begin{align*}
    \delta_i({-}) : \textbf{STLC} &\to \textbf{STLC}^\rho \\
    X &\mapsto j \mapsto
    \begin{cases}
      j = i \implies &X\\
      j \neq i \implies &1
    \end{cases}
  \end{align*}
  to be the function mapping an object $X \in \textbf{STLC}$ to a family
  $X_j$, whose $i$th component is $X$, and all other components are the
  terminal object.
\end{definition}
\begin{definition} Define
  \begin{align*}
    [{-}] : \textbf{STLC}^\rho &\to \textbf{STLC} \\
    (Y_j)_{j\in\rho} &\mapsto \prod_{j \in \rho} Y_j
  \end{align*}
  to be the function mapping a family $(Y_j)_{j\in\rho} \in \textbf{STLC}^\rho$
  to the product of its components.
\end{definition}
\vspace{0.5\baselineskip}
\noindent
With these definitions in hand, we can represent the type of a hypothetical
communication operation that communicates a global state $X \in
\textbf{STLC}^\rho$ from all processes to a single process $i \in \rho$ as
follows:
\[
  \textsf{com}_{X,i} : X \to \delta_i([X])
\]
\vspace{0.5\baselineskip}
\begin{definition}
  Let the \emph{category of synchronously interacting processes},
  $\textbf{SyncIntProc}_{\mathcal{C}} = \textbf{STLC}^\rho[\textsf{com}]$ be
  defined as the category $\textbf{STLC}^\rho$ of independent processes with
  freely adjoined arrows of the shape $\textsf{com}_{X,i} : X \to
  \delta_i([X])$ for any $X \in \textbf{STLC}^\rho$ and $i \in \rho$.
\end{definition}
\vspace{0.5\baselineskip}
\begin{theorem}
  Let $\mathcal{C}$ be a cartesian closed category. There is a covariant model
  of MTT${}_{\mathcal{M}^\rho_{\textrm{Chor}}}$ that has
  $\textbf{SyncIntProc}_{\mathcal{C}}$ as its compilation target category.
\end{theorem}
\vspace{0.5\baselineskip}
\noindent
In particular, in this model, only the transformation $\textsf{eval}_v :
(\square \fatsemi {}@ v) \Rightarrow \textsf{id}_{\circle}$ is built up from
the freely adjoined $\textsf{com}$ arrows, since it is the only one involving
communication. All other transformations are modeled by arrows already existing
as part of the cartesian closed structure of each processes' individual
category $\mathcal{C}$.
\begin{corollary}
  There is a translation function from the syntax category of
  MTT${}_{\mathcal{M}^\rho_{\textrm{Chor}}}$ to
  $\textbf{SyncIntProc}_{\mathcal{C}}$.
\end{corollary}
\vspace{0.5\baselineskip}
\noindent
In other words, this gives us a compilation procedure for Kami programs into
any target language with cartesian closed categorical semantics and synchronous
communication primitives.

\section*{Acknowledgements}
<<<<<<< HEAD
The Kami programming language\footnote{\url{https://nlnet.nl/project/Kami/}}
has been originally envisioned and preliminary research was carried out
together with Olivia R\"ohrig. The authors are tremendously grateful for
countless hours of debating syntax and semantics.
=======
The Kami programming language\footnote{\url{https://nlnet.nl/project/Kami/}} has been originally envisioned and preliminary research was
carried out together with Olivia R\"ohrig. The authors are tremendously
grateful for countless hours of debating syntax and semantics.
>>>>>>> 553f40c (Some changes:)

This project is funded through NGI Zero
Core\footnote{\url{https://nlnet.nl/core}}, a fund established by NLnet with
financial support from the European Commission's Next Generation Internet
program.

%%%%%%%%%%%%%%%%%%%%%%%%%%%%%%%%%%%%%%%%%%%%%%%%%%%%%%%%%%%%%%%%%%%%%%%%%%%%%%%%
\bibliographystyle{plainnat}
\bibliography{main}

\end{document}
